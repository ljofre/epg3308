\documentclass[]{article}
\usepackage{lmodern}
\usepackage{amssymb,amsmath}
\usepackage{ifxetex,ifluatex}
\usepackage{fixltx2e} % provides \textsubscript
\ifnum 0\ifxetex 1\fi\ifluatex 1\fi=0 % if pdftex
  \usepackage[T1]{fontenc}
  \usepackage[utf8]{inputenc}
\else % if luatex or xelatex
  \ifxetex
    \usepackage{mathspec}
  \else
    \usepackage{fontspec}
  \fi
  \defaultfontfeatures{Ligatures=TeX,Scale=MatchLowercase}
\fi
% use upquote if available, for straight quotes in verbatim environments
\IfFileExists{upquote.sty}{\usepackage{upquote}}{}
% use microtype if available
\IfFileExists{microtype.sty}{%
\usepackage{microtype}
\UseMicrotypeSet[protrusion]{basicmath} % disable protrusion for tt fonts
}{}
\usepackage[margin=1in]{geometry}
\usepackage{hyperref}
\hypersetup{unicode=true,
            pdftitle={Analisis Descriptivo},
            pdfauthor={Seomara Palominos - Leonardo Jofre},
            pdfborder={0 0 0},
            breaklinks=true}
\urlstyle{same}  % don't use monospace font for urls
\usepackage{graphicx,grffile}
\makeatletter
\def\maxwidth{\ifdim\Gin@nat@width>\linewidth\linewidth\else\Gin@nat@width\fi}
\def\maxheight{\ifdim\Gin@nat@height>\textheight\textheight\else\Gin@nat@height\fi}
\makeatother
% Scale images if necessary, so that they will not overflow the page
% margins by default, and it is still possible to overwrite the defaults
% using explicit options in \includegraphics[width, height, ...]{}
\setkeys{Gin}{width=\maxwidth,height=\maxheight,keepaspectratio}
\IfFileExists{parskip.sty}{%
\usepackage{parskip}
}{% else
\setlength{\parindent}{0pt}
\setlength{\parskip}{6pt plus 2pt minus 1pt}
}
\setlength{\emergencystretch}{3em}  % prevent overfull lines
\providecommand{\tightlist}{%
  \setlength{\itemsep}{0pt}\setlength{\parskip}{0pt}}
\setcounter{secnumdepth}{0}
% Redefines (sub)paragraphs to behave more like sections
\ifx\paragraph\undefined\else
\let\oldparagraph\paragraph
\renewcommand{\paragraph}[1]{\oldparagraph{#1}\mbox{}}
\fi
\ifx\subparagraph\undefined\else
\let\oldsubparagraph\subparagraph
\renewcommand{\subparagraph}[1]{\oldsubparagraph{#1}\mbox{}}
\fi

%%% Use protect on footnotes to avoid problems with footnotes in titles
\let\rmarkdownfootnote\footnote%
\def\footnote{\protect\rmarkdownfootnote}

%%% Change title format to be more compact
\usepackage{titling}

% Create subtitle command for use in maketitle
\newcommand{\subtitle}[1]{
  \posttitle{
    \begin{center}\large#1\end{center}
    }
}

\setlength{\droptitle}{-2em}
  \title{Analisis Descriptivo}
  \pretitle{\vspace{\droptitle}\centering\huge}
  \posttitle{\par}
  \author{Seomara Palominos - Leonardo Jofre}
  \preauthor{\centering\large\emph}
  \postauthor{\par}
  \predate{\centering\large\emph}
  \postdate{\par}
  \date{29 de junio de 2017}


\begin{document}
\maketitle

\subsection{Estadisticos Descriptivos}\label{estadisticos-descriptivos}

\begin{verbatim}
               n promedio  suma   std    CV asimetria curtosis minimo
Sepal.Length 150    5.843 876.5 0.828 0.142     0.309    2.394    4.3
             maximo  Q1  Q2  Q3
Sepal.Length    7.9 5.2 6.5 5.2
\end{verbatim}

\subsection{Matriz de Correlacion de
Pearson}\label{matriz-de-correlacion-de-pearson}

\begin{verbatim}
             Sepal.Length
Sepal.Length            1
\end{verbatim}

\subsection{Matriz de Correlacion de
Spearman}\label{matriz-de-correlacion-de-spearman}

\begin{verbatim}
             Sepal.Length Sepal.Width Petal.Length Petal.Width
Sepal.Length    1.0000000   0.7964289    0.9512612   0.9543464
Sepal.Width     0.7964289   1.0000000    0.7744356   0.7713321
Petal.Length    0.9512612   0.7744356    1.0000000   0.9767083
Petal.Width     0.9543464   0.7713321    0.9767083   1.0000000
\end{verbatim}

\pagebreak

\subsection{Histogramas}\label{histogramas}

\begin{verbatim}
[1] "Sepal.Length"
\end{verbatim}

\includegraphics{C:/Users/usuario/Desktop/cossa_files/figure-latex/unnamed-chunk-4-1.pdf}

\subsection{Graficas Box Plot}\label{graficas-box-plot}

\includegraphics{C:/Users/usuario/Desktop/cossa_files/figure-latex/unnamed-chunk-5-1.pdf}
\includegraphics{C:/Users/usuario/Desktop/cossa_files/figure-latex/unnamed-chunk-5-2.pdf}
\includegraphics{C:/Users/usuario/Desktop/cossa_files/figure-latex/unnamed-chunk-5-3.pdf}
\includegraphics{C:/Users/usuario/Desktop/cossa_files/figure-latex/unnamed-chunk-5-4.pdf}


\end{document}
